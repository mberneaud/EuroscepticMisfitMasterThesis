\begin{frame}{Overview}

\begin{enumerate}
\def\labelenumi{\arabic{enumi}.}
\tightlist
\item
  Motivation

  \begin{itemize}
  \tightlist
  \item
    The phenomenon of the ``Eurosceptic misfit''
  \item
    A gap in the research field
  \end{itemize}
\item
  Methodology

  \begin{itemize}
  \tightlist
  \item
    The data set
  \item
    The statistical models
  \end{itemize}
\item
  Findings

  \begin{itemize}
  \tightlist
  \item
    Eurosceptic misfit across time
  \item
    Eurosceptic misfit across regions
  \item
    Regression models
  \end{itemize}
\item
  Conclusion
\end{enumerate}

\end{frame}

\section{1. Motivation}\label{motivation}

\begin{frame}{The phenomenon of the ``Eurosceptic misfit''}

\begin{itemize}
\tightlist
\item
  Wide gap between popular levels of Euroscepticism and the aggregate
  vote share received by Eurosceptic parties
\item
  First mentioned by Taggart (1998)

  \begin{itemize}
  \tightlist
  \item
    He suspected the translation of Eurosceptic attitudes into
    Eurosceptic vote was driven by national contextual factors
  \end{itemize}
\end{itemize}

\textbf{Understanding national contextual drivers of the ``Eurosceptic
misfit'' is a puzzle-piece in understanding the political impact of
Eurosceptic parties}

\end{frame}

\begin{frame}{A gap in the research field}

\begin{itemize}
\tightlist
\item
  Literature on Euroscepticism mainly revolves around:

  \begin{itemize}
  \tightlist
  \item
    Categorisation of Euroscepticism
  \item
    Derterminants of Euroscepticism
  \end{itemize}
\item
  The Eurosceptic misfit has received no thorough scholarly attention
  outside a few mentions (e.g.~Taggart \& Sczerbiak, 2002; Verney
  (2011))
\end{itemize}

\textbf{Possibility to make a contribution to an underdeveloped research
field}

\end{frame}

\section{2. Methodology}\label{methodology}

\begin{frame}{2.1. The data set}

I created my own study data set from

\begin{itemize}
\item
  European Parliament election data from 1979 to 2009
\item
  Eurobarometer data adjacent to each election round
\item
  This yielded a data set which had

  \begin{itemize}
  \tightlist
  \item
    110 observations\\
  \item
    27 panel units\\
  \item
    7 time periods
  \end{itemize}
\end{itemize}

\end{frame}

\begin{frame}{2.2. The statistical models}

\begin{itemize}
\item
  \textbf{Dependent variable}: share of voters holding Eurosceptic
  attitudes - sum of the vote share Eurosceptic parties
\item
  \textbf{Independent variables}

  \begin{itemize}
  \tightlist
  \item
    Two measures of Eurosceptic attitudes
  \item
    Party system polarisation
  \item
    Effective number of parties
  \item
    Membership duration in the EU
  \item
    Location in Central and Eastern Europe
  \end{itemize}
\item
  Models

  \begin{itemize}
  \tightlist
  \item
    Fixed effects, random effects (and pooled OLS)
  \end{itemize}
\end{itemize}

\end{frame}

\section{3. Findings}\label{findings}

\begin{frame}{3.1. Eurosceptic misfit across time}

\begin{center}\includegraphics{graphs/graph-1} \end{center}

\end{frame}

\begin{frame}{3.2. Eurosceptic misfit across regions}

\begin{center}\includegraphics{graphs/displaying_plot-1} \end{center}

\end{frame}

\begin{frame}{3.3. Regression models}

\begin{table}[!htbp] \centering 
  \caption{Regression results} 
  \label{} 
\tiny 
\begin{tabular}{@{\extracolsep{5pt}}lcc} 
\\[-1.8ex]\hline 
\hline \\[-1.8ex] 
 & \multicolumn{2}{c}{\textit{Dependent variable:}} \\ 
\cline{2-3} 
\\[-1.8ex] & \multicolumn{2}{c}{Eurosceptic Misfit} \\ 
 & FE & RE \\ 
\\[-1.8ex] & (1) & (2)\\ 
\hline \\[-1.8ex] 
 General Euroscepticism & 0.48$^{***}$ & 0.45$^{***}$ \\ 
  & (0.10) & (0.09) \\ 
  & & \\ 
 Instrumental Euroscepticism & 0.38$^{***}$ & 0.43$^{***}$ \\ 
  & (0.06) & (0.05) \\ 
  & & \\ 
 Polarisation Index & $-$1.52$^{*}$ & $-$1.92$^{**}$ \\ 
  & (0.84) & (0.74) \\ 
  & & \\ 
 Effective Number of Parties & 0.02 & $-$0.06 \\ 
  & (0.12) & (0.10) \\ 
  & & \\ 
 Membership Duration & 0.08 & 0.04 \\ 
  & (0.08) & (0.06) \\ 
  & & \\ 
 Central/Eastern European &  & $-$5.52$^{*}$ \\ 
  &  & (3.09) \\ 
  & & \\ 
 Constant &  & 0.57 \\ 
  &  & (3.47) \\ 
  & & \\ 
\hline \\[-1.8ex] 
\hline 
\hline \\[-1.8ex] 
\textit{Note:}  & \multicolumn{2}{r}{$^{*}$p$<$0.1; $^{**}$p$<$0.05; $^{***}$p$<$0.01} \\ 
\end{tabular} 
\end{table}

\end{frame}

\begin{frame}{4. Conclusion}

\begin{itemize}
\tightlist
\item
  The Eurosceptic misfit has stayed relatively constant over time,
  except for a spike in 1994
\item
  Eurosceptic misfit somewhat smaller in CEE countries
\item
  Eurosceptic vote share generally fails to ``catch up'' with increases
  in popular Eurosceptic attitudes
\item
  Higher degrees of party polarisation shrink the Eurosceptic misfit
\end{itemize}

\end{frame}

\begin{frame}{Thank you for listening!}

Please check out my
\href{https://github.com/mberneaud/EuroscepticMisfitMasterThesis}{Github
repository} for all source codes for the analysis and presentation
documents:

\emph{github.com/mberneaud/EuroscepticMisfitMasterThesis}

\end{frame}

\begin{frame}{Extra: Model considerations}

\begin{itemize}
\tightlist
\item
  Combination of pooled OLS, fixed effects and random effects models
  used in the thesis

  \begin{itemize}
  \tightlist
  \item
    Fixed effects to cleanly isolate causal effects (read: removing
    unobserved heterogeneity) to measure country-specific variation
  \item
    Random effects to allow for inclusion of time-invariant variables
    (like region) and higher efficiency
  \item
    Pooled OLS for comparison
  \end{itemize}
\end{itemize}

\end{frame}
